%%%%%%%%%%%%%%%%%%%%%%%%%%%%%%%%%%%%%%%%%
% Medium Length Professional CV
% LaTeX Template
% Version 3.0 (December 17, 2022)
%
% This template originates from:
% https://www.LaTeXTemplates.com
%
% Author:
% Vel (vel@latextemplates.com)
%
% Original author:
% Trey Hunner (http://www.treyhunner.com/)
%
% License:
% CC BY-NC-SA 4.0 (https://creativecommons.org/licenses/by-nc-sa/4.0/)
%
%%%%%%%%%%%%%%%%%%%%%%%%%%%%%%%%%%%%%%%%%

%----------------------------------------------------------------------------------------
%	PACKAGES AND OTHER DOCUMENT CONFIGURATIONS
%----------------------------------------------------------------------------------------

\documentclass[
	%a4paper, % Uncomment for A4 paper size (default is US letter)
	11pt, % Default font size, can use 10pt, 11pt or 12pt
]{resume} % Use the resume class

\usepackage{ebgaramond} % Use the EB Garamond font
\usepackage{hyperref}

\newenvironment{myitemize}
  {\begin{itemize}
    \setlength{\itemsep}{-0.5em} \vspace{-0.5em}
  } % Reduce vertical spacing between items in the list for a tighter look
  {\end{itemize}}

%------------------------------------------------

\name{Chen Wen-Cheng} % Your name to appear at the top

% You can use the \address command up to 3 times for 3 different addresses or pieces of contact information
% Any new lines (\\) you use in the \address commands will be converted to symbols, so each address will appear as a single line.

% \address{iaiminwtg@gmail.com} % Contact information

%----------------------------------------------------------------------------------------

\begin{document}

%----------------------------------------------------------------------------------------
%	CONTACT INFO SECTION
%----------------------------------------------------------------------------------------
%
email: \href{mailto:iaiminwtg@gmail.com}{iaiminwtg@gmail.com}, Blog: \href{https://blog.errorbaker.tw/posts/cwc329}{https://blog.errorbaker.tw/posts/cwc329} \\
LinkedIn: \href{https://www.linkedin.com/in/wenchengchen715161143}{wenchengchen715161143}, GitHub: \href{https://github.com/cwc329}{cwc329}

%----------------------------------------------------------------------------------------
%	WORK EXPERIENCE SECTION
%----------------------------------------------------------------------------------------

\begin{rSection}{Objective}
    Experienced backend engineer with a strong ability to deliver functional and scalable solutions, having successfully implemented over 10 product features, with at least half completed independently.
    Proven track record of optimizing CI/CD processes, implementing efficient deployment strategies, and enhancing development environments.
    Seeking a senior backend engineer position to leverage expertise in database design, RESTful API implementation, CI/CD workflows, and Kubernetes management.
\end{rSection}

\begin{rSection}{Experience}

  I’ve been instrumental in \textbf{optimizing relational database schemas} and \textbf{implementing RESTful APIs} for two B2B SaaS products, collectively catering to hundreds of daily users.
  My contributions extend to spearheading \textbf{CI/CD workflows}, ensuring efficient \textbf{infrastructure management}, and overseeing the \textbf{effective maintenance of our Kubernetes infrastructure}.

	\begin{rSubsection}{PIMQ}{2021-02 - 2024-06}{Software Engineer}{Taipei, Taiwan}
    \item Backend Product Features Achievements:
    \begin{myitemize}
      \item Participated in \textbf{more than 10} feature implementation. 
      \item textbf{Independently} developed over \textbf{5 features}. Including:
      \begin{myitemize}
        \item Energy menagement system.
        \item Electricity comsumption and billing management system.
      \end{myitemize}
    \end{myitemize}
    \item CI/CD Optimization:
    \begin{myitemize}
      \item Reduced test coverage execution time by \textbf{25\%} through the implementation of a report cache.
      \item \textbf{Saved 80\% of CI/CD cost} by using on-demand runners.
      \item Achieved an \textbf{80\% reduction} in test site deployment time by leveraging AWS launch templates.
    \end{myitemize}
    \item Development Efficiency:
    \begin{myitemize}
      \item Introduced a development environment setup runbook, resulting in a \textbf{50\% time savings} for onboarding.
    \end{myitemize}
    \item Infrastructure and Monitoring:
    \begin{myitemize}
      \item Implemented container insight into EKS for effective monitoring of application logs.
      \item Leveraged AWS DynamoDB to simplify and streamline customized configuration deployment procedures.
    \end{myitemize}
    \item Collaborative Team Player:
    \begin{myitemize}
      \item Collaborated with colleagues to successfully \textbf{upgrade Kubernetes from 1.17 to 1.27} in EKS.
      \item \textbf{Led} the upgrade of the company’s MySQL from version \textbf{5.7 to 8 without downtime}.
    \end{myitemize}
  \end{rSubsection}

\end{rSection}

%----------------------------------------------------------------------------------------
%	TECHNICAL STRENGTHS SECTION
%----------------------------------------------------------------------------------------

\begin{rSection}{Technical Strengths}

	\begin{tabular}{@{} >{\bfseries}l @{\hspace{6ex}} l @{}}
    Programming Languages:& Javascript, Shell scripts \\
    Database Technologies:& MySQL, Redis, DynamoDB \\
    Cloud Platforms:& AWS (EC2, DynamoDB, EKS) \\
    Containerization:& Docker, Kubernetes \\
    IaC Tools:& AWS CDK, Terraform \\
    CI/CD Tools:& GitHub actions \\
	\end{tabular}

\end{rSection}

%----------------------------------------------------------------------------------------
%	EDUCATION SECTION
%----------------------------------------------------------------------------------------

\begin{rSection}{Education}
	
	\textbf{National Taiwan University, Taipei} \hfill \textit{June 2015} \\ 
	B.A. in Philosophy

\end{rSection}

%----------------------------------------------------------------------------------------
%	EXAMPLE SECTION
%----------------------------------------------------------------------------------------

%\begin{rSection}{Section Name}

	%Section content\ldots

%\end{rSection}

%----------------------------------------------------------------------------------------

\end{document}
